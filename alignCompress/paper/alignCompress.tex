\documentclass[a4paper,10pt,oneside]{article}
 
\usepackage{times}
\usepackage{epsfig}
 
\usepackage{verbatim}
 
\usepackage[none,bottom,dark,english]{draftcopy}
\draftcopyName{ DRAFT \number\day\/\number\month\/\number\year}
 
\usepackage[verbose,a4paper]{geometry}
%\geometry{top=2cm,bottom=2cm,left=1.5cm,right=1.5cm,nohead}
%\geometry{top=1.5cm,bottom=1.5cm,left=4cm,right=1.5cm,nohead}




\begin{document}

\title{Aligning Compressible Sequences.}

\author{
  David R. Powell, Lloyd Allison and Trevor I. Dix \\
%  Department of Computer Science, \\
%  Monash University, \\
%  Australia 3168.
}

\date{}
\maketitle


\section{ROC Plots}

For the ROC plots, first generated 10 parent sequences of the form $gen(100)
+ subseq(100) + gen(200)$.  For each parent children were generated having
only the subsequence in common with differing number of mutations.  Every parent
had 20 children generated; 4 with 10 mutations, 4 with 30 mutations, 4 with 50
mutations, 4 with 80 mutations and 4 with 100 mutations.  Each parent was then
tested against all children.

\begin{figure}[htb]
\centerline{\epsfig{file=p1.eps,width=0.7\textwidth}}
%\caption{\label{fig:} The caption}
\end{figure}

\begin{figure}[htb]
\centerline{\epsfig{file=p2.eps,width=0.7\textwidth}}
%\caption{\label{fig:} The caption}
\end{figure}

\begin{figure}[htb]
\centerline{\epsfig{file=p3.eps,width=0.7\textwidth}}
%\caption{\label{fig:} The caption}
\end{figure}

\begin{figure}[htb]
\centerline{\epsfig{file=p4.eps,width=0.7\textwidth}}
%\caption{\label{fig:} The caption}
\end{figure}

\begin{figure}[htb]
\centerline{\epsfig{file=p5.eps,width=0.7\textwidth}}
%\caption{\label{fig:} The caption}
\end{figure}

\clearpage

\section{Log-odds plots}

The first sequence was generated of the form $gen(100) + subseq(100) +
gen(200)$.  The second of the form $gen(200) + mutate(subseq, ntimes) +
gen(100)$.  The number of mutations `ntimes' was varied between 0 and 300.
The two sequences were then compared with our program and with prss.  For our
program we simply plot the log-odds ratio.  For the prss program we plot
$-log(p/(1-p))$.  This seems a reasonable comparison.  The 0 line is a natural
cut-off for the log-odds comparison, but the PRSS program is typically used
with a cutoff around $10^-3$ which corresponds to about 9 on our graphs.
Each data point on the graph is the average of 10 such comparisons.

\begin{figure}[htb]
\centerline{\epsfig{file=r1.eps,width=0.7\textwidth}}
%\caption{\label{fig:} The caption}
\end{figure}

\begin{figure}[htb]
\centerline{\epsfig{file=r2.eps,width=0.7\textwidth}}
%\caption{\label{fig:} The caption}
\end{figure}

\begin{figure}[htb]
\centerline{\epsfig{file=r3.eps,width=0.7\textwidth}}
%\caption{\label{fig:} The caption}
\end{figure}




\end{document} 
