\documentclass[a4paper,10pt,oneside]{article}
 
\usepackage{times}
\usepackage{epsfig}
 
\usepackage{verbatim}
 
\usepackage[none,bottom,dark,english]{draftcopy}
\draftcopyName{ DRAFT \number\day\/\number\month\/\number\year}
 
\usepackage[verbose,a4paper]{geometry}
%\geometry{top=2cm,bottom=2cm,left=1.5cm,right=1.5cm,nohead}
%\geometry{top=1.5cm,bottom=1.5cm,left=4cm,right=1.5cm,nohead}


\begin{document}

\title{Aligning Compressible Sequences.}

\author{
  David R. Powell, Lloyd Allison and Trevor I. Dix \\
%  Department of Computer Science, \\
%  Monash University, \\
%  Australia 3168.
}

\date{}
\maketitle


\section{ROC Plots}

For the ROC plots, first generated 10 parent sequences of the form
$gen(50+-50) + subseq(120+-30) + gen(100+-50)$.  In the case where 2 models
are used, the first model is used for 5 parent sequences and the second model
for 5 parent sequences.  For each parent, children were generated having only
the subsequence in common with differing number of mutations.  Every parent
had 20 children generated; 4 with 10 mutations, 4 with 30 mutations, 4 with 50
mutations, 4 with 60 mutations and 4 with 80 mutations.  Each parent was then
tested against all children.  Each child was of the form $gen(100+-50) +
mutated_subseq + gen(50+-50)$.


Table for Fig~\ref{fig:1}. Error rate 1\%.
\begin{tabular}{|l|c|c|c|c|} \hline
Algorithm & true positives & possible true positives & v1 & v2 \\ \hline
prss & 191 & 200 & 0.0148499 & 0.0149571 \\
S-W score & 191 & 200 & 82 & 81 \\
Optimal alignment & 179 & 200 & -4.78197 & -4.80126 \\
Summed alignments & 194 & 200 & -0.075325 & -0.086438 \\\hline
\end{tabular}

Table for Fig~\ref{fig:2}. Error rate 1\%.
\begin{tabular}{|l|c|c|c|c|} \hline
Algorithm & true positives & possible true positives & v1 & v2 \\ \hline
prss & 177 & 200 & 0.0141588 & 0.0148322 \\
S-W score & 166 & 200 & 186 & 186 \\
Optimal alignment & 147 & 200 & -5.5386 & -5.58365 \\
Summed alignments & 180 & 200 & -0.628439 & -0.631091 \\\hline
\end{tabular}

Table for Fig~\ref{fig:3}. Error rate 1\%.
\begin{tabular}{|l|c|c|c|c|} \hline
Algorithm & true positives & possible true positives & v1 & v2 \\ \hline
prss & 156 & 200 & 9.81967e-08 & 1.06702e-07 \\
S-W score & 146 & 200 & 207 & 206 \\
Optimal alignment & 144 & 200 & -5.20987 & -5.2934 \\
Summed alignments & 177 & 200 & -1.69442 & -1.73031 \\\hline
\end{tabular}

Table for Fig~\ref{fig:4}. Error rate 1\%.
\begin{tabular}{|l|c|c|c|c|} \hline
Algorithm & true positives & possible true positives & v1 & v2 \\ \hline
prss & 176 & 200 & 0.0153629 & 0.01537 \\
S-W score & 139 & 200 & 186 & 186 \\
Optimal alignment & 145 & 200 & -5.4425 & -5.50113 \\
Summed alignments & 176 & 200 & -0.745922 & -0.951519 \\\hline
\end{tabular}

Table for Fig~\ref{fig:5}. Error rate 1\%.
\begin{tabular}{|l|c|c|c|c|} \hline
Algorithm & true positives & possible true positives & v1 & v2 \\ \hline
prss & 175 & 200 & 0.0153587 & 0.0156386 \\
S-W score & 156 & 200 & 189 & 189 \\
Optimal alignment & 147 & 200 & -4.94089 & -4.96453 \\
Summed alignments & 180 & 200 & -0.914275 & -1.00804 \\\hline
\end{tabular}

Table for Fig~\ref{fig:6}. Error rate 1\%.
\begin{tabular}{|l|c|c|c|c|} \hline
Algorithm & true positives & possible true positives & v1 & v2 \\ \hline
prss & 158 & 200 & 2.17295e-05 & 2.20386e-05 \\
S-W score & 140 & 200 & 246 & 246 \\
Optimal alignment & 174 & 200 & 3.08782 & 3.05955 \\
Summed alignments & 184 & 200 & -1.08664 & -1.11433 \\\hline
\end{tabular}

Table for Fig~\ref{fig:7}. Error rate 1\%.
\begin{tabular}{|l|c|c|c|c|} \hline
Algorithm & true positives & possible true positives & v1 & v2 \\ \hline
prss & 209 & 250 & 1.28192e-07 & 1.66944e-07 \\
S-W score & 193 & 250 & 218 & 217 \\
Optimal alignment & 234 & 250 & 10.8823 & 10.7208 \\
Summed alignments & 245 & 250 & 2.12835 & 2.06153 \\\hline
\end{tabular}





For a given cutoff of 0 for our algorithms, and $10^{-5}$ for prss (raw S-W
score not shown).  The following tables summarise the number of correct
positives and incorrect positives.

Table for Fig~\ref{fig:1}.
\begin{tabular}{|l||c|c|c|c|} \hline
Algorithm & correct positives & incorrect positives & total related & total unrelated \\ \hline
prss & 176 & 0 & 200 & 1800 \\ 
Optimal alignment & 165 & 1 & 200 & 1800 \\ 
Summed alignments & 193 & 16 & 200 & 1800 \\ 
\hline \end{tabular}

Table for Fig~\ref{fig:2}.
\begin{tabular}{|l||c|c|c|c|} \hline
Algorithm & correct positives & incorrect positives & total related & total unrelated \\ \hline
prss & 99 & 0 & 200 & 1800 \\ 
Optimal alignment & 119 & 1 & 200 & 1800 \\ 
Summed alignments & 179 & 10 & 200 & 1800 \\ 
\hline \end{tabular}

Table for Fig~\ref{fig:3}.
\begin{tabular}{|l||c|c|c|c|} \hline
Algorithm & correct positives & incorrect positives & total related & total unrelated \\ \hline
prss & 167 & 118 & 200 & 1799 \\ 
Optimal alignment & 120 & 0 & 200 & 1799 \\ 
Summed alignments & 174 & 1 & 200 & 1799 \\ 
\hline \end{tabular}

Table for Fig~\ref{fig:4}.
\begin{tabular}{|l||c|c|c|c|} \hline
Algorithm & correct positives & incorrect positives & total related & total unrelated \\ \hline
prss & 121 & 0 & 200 & 1800 \\ 
Optimal alignment & 128 & 0 & 200 & 1800 \\ 
Summed alignments & 172 & 11 & 200 & 1800 \\ 
\hline \end{tabular}

Table for Fig~\ref{fig:5}.
\begin{tabular}{|l||c|c|c|c|} \hline
Algorithm & correct positives & incorrect positives & total related & total unrelated \\ \hline
prss & 117 & 0 & 200 & 1800 \\ 
Optimal alignment & 125 & 0 & 200 & 1800 \\ 
Summed alignments & 173 & 10 & 200 & 1800 \\ 
\hline \end{tabular}

Table for Fig~\ref{fig:6}.
\begin{tabular}{|l||c|c|c|c|} \hline
Algorithm & correct positives & incorrect positives & total related & total unrelated \\ \hline
prss & 156 & 8 & 200 & 1800 \\ 
1st MM & 185 & 126 & 200 & 1800 \\ 
Blend model & 182 & 10 & 200 & 1800 \\ 
\hline \end{tabular}

Table for Fig~\ref{fig:7}.
\begin{tabular}{|l||c|c|c|c|} \hline
Algorithm & correct positives & incorrect positives & total related & total unrelated \\ \hline
prss & 224 & 135 & 250 & 2250 \\ 
1st MM  & 248 & 957 & 250 & 2250 \\ 
Blend model & 246 & 116 & 250 & 2250 \\ 
\hline \end{tabular}




\begin{figure}[htb]
\centerline{\epsfig{file=p1.eps,width=0.7\textwidth}}
\caption{\label{fig:1}}
\end{figure}

\begin{figure}[htb]
\centerline{\epsfig{file=p2.eps,width=0.7\textwidth}}
\caption{\label{fig:2}}
\end{figure}

\begin{figure}[htb]
\centerline{\epsfig{file=p3.eps,width=0.7\textwidth}}
\caption{\label{fig:3}}
\end{figure}

\begin{figure}[htb]
\centerline{\epsfig{file=p4.eps,width=0.7\textwidth}}
\caption{\label{fig:4}}
\end{figure}

\begin{figure}[htb]
\centerline{\epsfig{file=p5.eps,width=0.7\textwidth}}
\caption{\label{fig:5}}
\end{figure}

\begin{figure}[htb]
\centerline{\epsfig{file=p6.eps,width=0.7\textwidth}}
\caption{\label{fig:6}}
\end{figure}

\begin{figure}[htb]
\centerline{\epsfig{file=p7.eps,width=0.7\textwidth}}
\caption{\label{fig:7}}
\end{figure}




\begin{comment}
\clearpage

\section{Log-odds plots}

The first sequence was generated of the form $gen(100) + subseq(100) +
gen(200)$.  The second of the form $gen(200) + mutate(subseq, ntimes) +
gen(100)$.  The number of mutations `ntimes' was varied between 0 and 300.
The two sequences were then compared with our program and with prss.  For our
program we simply plot the log-odds ratio.  For the prss program we plot
$-log(p)$.  This seems a reasonable comparison.  The 0 line is a natural
cut-off for the log-odds comparison, but the PRSS program is typically used
with a cutoff around $10^{-3}$ which corresponds to about 10 on our graphs.
Each data point on the graph is the average of 10 such comparisons.

\begin{figure}[htb]
\centerline{\epsfig{file=r1.eps,width=0.7\textwidth}}
%\caption{\label{fig:} The caption}
\end{figure}

\begin{figure}[htb]
\centerline{\epsfig{file=r2.eps,width=0.7\textwidth}}
%\caption{\label{fig:} The caption}
\end{figure}

\begin{figure}[htb]
\centerline{\epsfig{file=r3.eps,width=0.7\textwidth}}
%\caption{\label{fig:} The caption}
\end{figure}

\begin{figure}[htb]
\centerline{\epsfig{file=r4.eps,width=0.7\textwidth}}
%\caption{\label{fig:} The caption}
\end{figure}
\end{comment}


\end{document} 
